\documentclass[DM,lsstdraft,STR,toc]{lsstdoc}
\usepackage{geometry}
\usepackage{longtable,booktabs}
\usepackage{enumitem}
\usepackage{arydshln}

\input meta.tex

\providecommand{\tightlist}{
  \setlength{\itemsep}{0pt}\setlength{\parskip}{0pt}}

\begin{document}

\def\milestoneName{Fall 2019 Pipelines Release Acceptance Test Campaign}
\def\milestoneId{LVV-P65}
\def\product{Acceptance}

\setDocCompact{true}

\title{ LVV-P65 Fall 2019 Pipelines Release Acceptance Test Campaign Test Plan and Report}
\setDocRef{\lsstDocType-\lsstDocNum}
\date{\vcsdate}
\setDocUpstreamLocation{\url{https://github.com/lsst/lsst-texmf/examples}}
\author{ Jeffrey Carlin }

\input history_and_info.tex


\setDocAbstract{
This is the test plan and report for LVV-P65 (Fall 2019 Pipelines Release Acceptance Test Campaign),
an LSST level 2 milestone pertaining to the Data Management Subsystem.
}


\maketitle

\section{Introduction}
\label{sect:intro}


\subsection{Objectives}
\label{sect:objectives}

This Acceptance Test campaign aims to verify a small number of
\href{https://lse-61.lsst.io/}{DMSR} (\citeds{LSE-61}) requirements related to
the LSST Science Pipelines. It will be executed in conjunction with the
release of Science Pipelines Version 19.0.0, but the pipeline release is
not contingent upon this test campaign. This Test Plan aims to
demonstrate that the included requirements have been met by Version
19.0.0 of the Pipelines, and to thus fully verify their completion and
readiness for LSST Operations.



\subsection{System Overview}
\label{sect:systemoverview}

The tests to be executed are intended to verify that the DM system
satisfies a subset of the requirements outlined in the Data Management
System Requirements (DMSR; \href{https://lse-61.lsst.io/}{\citeds{LSE-61}} ).
This subset of requirements is related to pipeline algorithms, and was
selected for this campaign to coincide with the release of a new version
of the LSST Science Pipelines. Additional DMSR requirements will be
verified in later Acceptance Test
Campaigns.\\[2\baselineskip]\textbf{Applicable
Documents:}\\[2\baselineskip]\citeds{LSE-61} Data Management System
Requirements\\
\citeds{LDM-503} Data Management Test Plan\\[2\baselineskip]The tests will be
performed using the HSC-RC2 dataset (as defined in
\href{https://jira.lsstcorp.org/browse/DM-11345}{DM-11345} ). When
possible, we will start our tests with the data products resulting from
processing HSC-RC2 with the w\_2019\_46 weekly pipelines release (
\href{https://jira.lsstcorp.org/browse/DM-22223}{DM-22223} ) that was
used to create v19 of the Science Pipelines.~


\subsection{Document Overview}
\label{sect:docoverview}

This document was generated from Jira, obtaining the relevant information from the 
\href{https://jira.lsstcorp.org/secure/Tests.jspa#/testPlan/LVV-P65}{LVV-P65}
~Jira Test Plan and related Test Cycles (
  \href{https://jira.lsstcorp.org/secure/Tests.jspa#/testCycle/LVV-C115}{LVV-C115}
).

Section \ref{sect:intro} provides an overview of the test campaign, the system under test (\product{}),
the applicable documentation, and explains how this document is organized.
Section \ref{sect:configuration}  describes the configuration used for this test.
Section \ref{sect:personnel} describes the necessary roles and lists the individuals assigned to them.
%Section \ref{sect:plannedtestactivities} provides the list of planned test cycles and test cases,
including all relevant information that fully describes the test campaign.

Section \ref{sect:overview} provides a summary of the test results, including an overview in Table \ref{table:summary},
an overall assessment statement and suggestions for possible improvements.
Section \ref{sect:detailedtestresults} provides detailed results for each step in each test case.

The current status of test plan LVV-P65 in Jira is \textbf{ Draft }.

\subsection{References}
\label{sect:references}
\renewcommand{\refname}{}
\bibliography{lsst,refs,books,refs_ads}
\section{Test Configuration}
\label{sect:configuration}

\subsection{Data Collection}

  Observing is not required for this test campaign.

\subsection{Verification Environment}
\label{sect:hwconf}
  The ``lsst-lsp-stable'' instance of the LSST Science Platform (LSP),
hosted at the LDF, and the ``lsst-dev'' development cluster at NCSA.~In
particular, we will use Release 19.0.0 of the Pipelines, whose release
is DM Milestone LDM-503-11b (Test Plan located
\href{https://jira.lsstcorp.org/secure/Tests.jspa\#/testPlan/LVV-P62}{here)}.


  \subsection{Entry Criteria}
  Release and availability of Science Pipelines version 19.




\newpage
\section{Personnel}
\label{sect:personnel}

The personnel involved in the test campaign is shown in the following table.

\begin{longtable}{p{3cm}p{3cm}p{3cm}p{6cm}}
\hline
\multicolumn{2}{r}{Test Plan (LVV-P65) owner:} &
\multicolumn{2}{l}{\textbf{ Jeffrey Carlin } }\\\hline
\multicolumn{2}{r}{ LVV-C115 owner:} &
\multicolumn{2}{l}{\textbf{
    Jeffrey Carlin
}
} \\\hline
\textbf{Test Case} & \textbf{Assigned to} & \textbf{Executed by} & \textbf{Additional Test Personnel} \\ \hline
\href{https://jira.lsstcorp.org/secure/Tests.jspa#/testCase/LVV-T41}{LVV-T41}
& {\small Jim Bosch } & {\small  } &
\begin{minipage}[]{6cm}
\smallskip
{\small  }
\medskip
\end{minipage}
\\ \hline
\href{https://jira.lsstcorp.org/secure/Tests.jspa#/testCase/LVV-T62}{LVV-T62}
& {\small Jim Bosch } & {\small  } &
\begin{minipage}[]{6cm}
\smallskip
{\small  }
\medskip
\end{minipage}
\\ \hline
\href{https://jira.lsstcorp.org/secure/Tests.jspa#/testCase/LVV-T132}{LVV-T132}
& {\small Robert Gruendl } & {\small  } &
\begin{minipage}[]{6cm}
\smallskip
{\small  }
\medskip
\end{minipage}
\\ \hline
\href{https://jira.lsstcorp.org/secure/Tests.jspa#/testCase/LVV-T40}{LVV-T40}
& {\small Jim Bosch } & {\small  } &
\begin{minipage}[]{6cm}
\smallskip
{\small  }
\medskip
\end{minipage}
\\ \hline
\href{https://jira.lsstcorp.org/secure/Tests.jspa#/testCase/LVV-T1240}{LVV-T1240}
& {\small Jim Bosch } & {\small  } &
\begin{minipage}[]{6cm}
\smallskip
{\small  }
\medskip
\end{minipage}
\\ \hline
\href{https://jira.lsstcorp.org/secure/Tests.jspa#/testCase/LVV-T376}{LVV-T376}
& {\small Leanne Guy } & {\small  } &
\begin{minipage}[]{6cm}
\smallskip
{\small  }
\medskip
\end{minipage}
\\ \hline
\href{https://jira.lsstcorp.org/secure/Tests.jspa#/testCase/LVV-T378}{LVV-T378}
& {\small Leanne Guy } & {\small  } &
\begin{minipage}[]{6cm}
\smallskip
{\small  }
\medskip
\end{minipage}
\\ \hline
\href{https://jira.lsstcorp.org/secure/Tests.jspa#/testCase/LVV-T377}{LVV-T377}
& {\small Leanne Guy } & {\small  } &
\begin{minipage}[]{6cm}
\smallskip
{\small  }
\medskip
\end{minipage}
\\ \hline
\href{https://jira.lsstcorp.org/secure/Tests.jspa#/testCase/LVV-T28}{LVV-T28}
& {\small Colin Slater } & {\small  } &
\begin{minipage}[]{6cm}
\smallskip
{\small  }
\medskip
\end{minipage}
\\ \hline
\href{https://jira.lsstcorp.org/secure/Tests.jspa#/testCase/LVV-T43}{LVV-T43}
& {\small Jim Bosch } & {\small  } &
\begin{minipage}[]{6cm}
\smallskip
{\small  }
\medskip
\end{minipage}
\\ \hline
\end{longtable}

\newpage

\section{Test Campaign Overview}
\label{sect:overview}

\subsection{Summary}
\label{sect:summarytable}

\begin{longtable}{p{2cm}p{2.5cm}p{9cm}p{2.5cm}}
\toprule
\multicolumn{3}{l}{ Test Plan {\bf LVV-P65: Fall 2019 Pipelines Release Acceptance Test Campaign
 }} & Draft \\\hline

  \multicolumn{3}{l}{ Test Cycle {\bf LVV-C115: Fall 2019 Pipelines Release Acceptance Test Campaign
 }} & Not Executed \\\hline

  {\bf \footnotesize test case} & {\bf \footnotesize status} & {\bf \footnotesize comment} & {\bf \footnotesize issues} \\\toprule

    \href{https://jira.lsstcorp.org/secure/Tests.jspa#/testCase/LVV-T41}{LVV-T41}
    & Not Executed &
    \begin{minipage}[]{9cm}
    \smallskip
    
    \medskip
    \end{minipage}
    &
    \\\hline
    \href{https://jira.lsstcorp.org/secure/Tests.jspa#/testCase/LVV-T62}{LVV-T62}
    & Not Executed &
    \begin{minipage}[]{9cm}
    \smallskip
    
    \medskip
    \end{minipage}
    &
    \\\hline
    \href{https://jira.lsstcorp.org/secure/Tests.jspa#/testCase/LVV-T132}{LVV-T132}
    & Not Executed &
    \begin{minipage}[]{9cm}
    \smallskip
    
    \medskip
    \end{minipage}
    &
    \\\hline
    \href{https://jira.lsstcorp.org/secure/Tests.jspa#/testCase/LVV-T40}{LVV-T40}
    & Not Executed &
    \begin{minipage}[]{9cm}
    \smallskip
    
    \medskip
    \end{minipage}
    &
    \\\hline
    \href{https://jira.lsstcorp.org/secure/Tests.jspa#/testCase/LVV-T1240}{LVV-T1240}
    & Not Executed &
    \begin{minipage}[]{9cm}
    \smallskip
    
    \medskip
    \end{minipage}
    &
    \\\hline
    \href{https://jira.lsstcorp.org/secure/Tests.jspa#/testCase/LVV-T376}{LVV-T376}
    & Not Executed &
    \begin{minipage}[]{9cm}
    \smallskip
    
    \medskip
    \end{minipage}
    &
    \\\hline
    \href{https://jira.lsstcorp.org/secure/Tests.jspa#/testCase/LVV-T378}{LVV-T378}
    & Not Executed &
    \begin{minipage}[]{9cm}
    \smallskip
    
    \medskip
    \end{minipage}
    &
    \\\hline
    \href{https://jira.lsstcorp.org/secure/Tests.jspa#/testCase/LVV-T377}{LVV-T377}
    & Not Executed &
    \begin{minipage}[]{9cm}
    \smallskip
    
    \medskip
    \end{minipage}
    &
    \\\hline
    \href{https://jira.lsstcorp.org/secure/Tests.jspa#/testCase/LVV-T28}{LVV-T28}
    & Not Executed &
    \begin{minipage}[]{9cm}
    \smallskip
    
    \medskip
    \end{minipage}
    &
    \\\hline
    \href{https://jira.lsstcorp.org/secure/Tests.jspa#/testCase/LVV-T43}{LVV-T43}
    & Not Executed &
    \begin{minipage}[]{9cm}
    \smallskip
    
    \medskip
    \end{minipage}
    &
    \\\hline
\caption{Test Results Summary}
\label{table:summary}
\end{longtable}

\subsection{Overall Assessment}
\label{sect:overallassessment}

Not yet available.

\subsection{Recommended Improvements}
\label{sect:recommendations}

Not yet available.

\newpage
\section{Detailed Test Results}
\label{sect:detailedtestresults}

\subsection{Test Cycle LVV-C115 }

Open test cycle {\it \href{https://jira.lsstcorp.org/secure/Tests.jspa#/testrun/LVV-C115}{Fall 2019 Pipelines Release Acceptance Test Campaign
}} in Jira.

Fall 2019 Pipelines Release Acceptance Test Campaign
\\
Status: Not Executed

This test cycle verifies a subset of
\href{https://lse-61.lsst.io/}{DMSR} (\citeds{LSE-61}) requirements related to
the LSST Science Pipelines, in order to verify their completion and
readiness for LSST Operations (i.e., that the requirements laid out in
\citeds{LSE-61} have been met by the DM Systems).


\subsubsection{Software Version/Baseline}
All tests will be performed with LSST Science Pipelines release version
19.0.0, including its algorithms and resulting science data products.~


\subsubsection{Configuration}
Not provided.

\subsubsection{Test Cases in LVV-C115 Test Cycle}

\paragraph{Test Case LVV-T41 - Verify implementation of Generate PSF for Visit Images
 }\mbox{}\\

Open  \href{https://jira.lsstcorp.org/secure/Tests.jspa#/testCase/LVV-T41}{\textit{ LVV-T41 } }
test case in Jira.

Verify that Processed Visit Images produced by the DRP and AP pipelines
are associated with a model from which one can obtain an image of the
PSF given a point on the image.


\textbf{ Preconditions}:\\


Execution status: {\bf Not Executed }

Final comment:\\


Detailed steps results:

\begin{longtable}{p{1cm}p{15cm}}
\hline
{Step} & Step Details\\ \hline
1 & Description \\
 & \begin{minipage}[t]{15cm}
{\footnotesize
\smallskip
Identify a dataset with processed visit images in multiple filters.

\medskip }
\end{minipage}
\\ \cdashline{2-2}


 & Expected Result \\
 & \begin{minipage}[t]{15cm}{\footnotesize
\smallskip

\medskip }
\end{minipage} \\ \cdashline{2-2}

 & Actual Result \\
 & \begin{minipage}[t]{15cm}{\footnotesize
\smallskip

\medskip }
\end{minipage} \\ \cdashline{2-2}

 & Status: \textbf{ Not Executed } \\ \hline

2 & Description \\
 & \begin{minipage}[t]{15cm}
{\footnotesize
\smallskip
Identify the path to the data repository, which we will refer to as
`DATA/path', then execute the following:

\medskip }
\end{minipage}
\\ \cdashline{2-2}


 & Expected Result \\
 & \begin{minipage}[t]{15cm}{\footnotesize
\smallskip
Butler repo available for reading.

\medskip }
\end{minipage} \\ \cdashline{2-2}

 & Actual Result \\
 & \begin{minipage}[t]{15cm}{\footnotesize
\smallskip

\medskip }
\end{minipage} \\ \cdashline{2-2}

 & Status: \textbf{ Not Executed } \\ \hline

3 & Description \\
 & \begin{minipage}[t]{15cm}
{\footnotesize
\smallskip
Select Objects classified as point sources on at least 10 different
processed visit images (including all bands). ~Evaluate the PSF model at
the positions of these Objects, and verify that subtracting a scaled
version of the PSF model from the processed visit image yields residuals
consistent with pure noise.

\medskip }
\end{minipage}
\\ \cdashline{2-2}


 & Expected Result \\
 & \begin{minipage}[t]{15cm}{\footnotesize
\smallskip
Images with the PSF model subtracted, leaving only residuals that are
consistent with being noise.

\medskip }
\end{minipage} \\ \cdashline{2-2}

 & Actual Result \\
 & \begin{minipage}[t]{15cm}{\footnotesize
\smallskip

\medskip }
\end{minipage} \\ \cdashline{2-2}

 & Status: \textbf{ Not Executed } \\ \hline

\end{longtable}

\paragraph{Test Case LVV-T62 - Verify implementation of Provide PSF for Coadded Images
 }\mbox{}\\

Open  \href{https://jira.lsstcorp.org/secure/Tests.jspa#/testCase/LVV-T62}{\textit{ LVV-T62 } }
test case in Jira.

Verify that all coadd images produced by the DRP pipelines include a
model from which an image of the PSF at any point on the coadd can be
obtained.


\textbf{ Preconditions}:\\
Fully covered by preconditions for
\href{https://jira.lsstcorp.org/secure/Tests.jspa\#/testCase/LVV-T16}{LVV-T16}.


Execution status: {\bf Not Executed }

Final comment:\\


Detailed steps results:

\begin{longtable}{p{1cm}p{15cm}}
\hline
{Step} & Step Details\\ \hline
1 & Description \\
 & \begin{minipage}[t]{15cm}
{\footnotesize
\smallskip
The DM Stack shall be initialized using the loadLSST script (as
described in LVV-T10 - DRP-00-00)

\medskip }
\end{minipage}
\\ \cdashline{2-2}


 & Expected Result \\
 & \begin{minipage}[t]{15cm}{\footnotesize
\smallskip

\medskip }
\end{minipage} \\ \cdashline{2-2}

 & Actual Result \\
 & \begin{minipage}[t]{15cm}{\footnotesize
\smallskip

\medskip }
\end{minipage} \\ \cdashline{2-2}

 & Status: \textbf{ Not Executed } \\ \hline

2 & Description \\
 & \begin{minipage}[t]{15cm}
{\footnotesize
\smallskip
A ``Data Butler'' will be initialized to access the repository.

\medskip }
\end{minipage}
\\ \cdashline{2-2}


 & Expected Result \\
 & \begin{minipage}[t]{15cm}{\footnotesize
\smallskip

\medskip }
\end{minipage} \\ \cdashline{2-2}

 & Actual Result \\
 & \begin{minipage}[t]{15cm}{\footnotesize
\smallskip

\medskip }
\end{minipage} \\ \cdashline{2-2}

 & Status: \textbf{ Not Executed } \\ \hline

3 & Description \\
 & \begin{minipage}[t]{15cm}
{\footnotesize
\smallskip
For each combination of tract/patch/filter, the PVI will be retrieved
from the Butler, and the existence of all components described in Test
items section §4.6.2 will be verified.

\medskip }
\end{minipage}
\\ \cdashline{2-2}


 & Expected Result \\
 & \begin{minipage}[t]{15cm}{\footnotesize
\smallskip

\medskip }
\end{minipage} \\ \cdashline{2-2}

 & Actual Result \\
 & \begin{minipage}[t]{15cm}{\footnotesize
\smallskip

\medskip }
\end{minipage} \\ \cdashline{2-2}

 & Status: \textbf{ Not Executed } \\ \hline

4 & Description \\
 & \begin{minipage}[t]{15cm}
{\footnotesize
\smallskip
Scripts from the pipe\_analysis package will be run on every visit to
check for the presence of data products and make plots

\medskip }
\end{minipage}
\\ \cdashline{2-2}


 & Expected Result \\
 & \begin{minipage}[t]{15cm}{\footnotesize
\smallskip

\medskip }
\end{minipage} \\ \cdashline{2-2}

 & Actual Result \\
 & \begin{minipage}[t]{15cm}{\footnotesize
\smallskip

\medskip }
\end{minipage} \\ \cdashline{2-2}

 & Status: \textbf{ Not Executed } \\ \hline

5 & Description \\
 & \begin{minipage}[t]{15cm}
{\footnotesize
\smallskip
Ten patches will be chosen at random and inspected by eye for unmasked
artifacts.

\medskip }
\end{minipage}
\\ \cdashline{2-2}


 & Expected Result \\
 & \begin{minipage}[t]{15cm}{\footnotesize
\smallskip

\medskip }
\end{minipage} \\ \cdashline{2-2}

 & Actual Result \\
 & \begin{minipage}[t]{15cm}{\footnotesize
\smallskip

\medskip }
\end{minipage} \\ \cdashline{2-2}

 & Status: \textbf{ Not Executed } \\ \hline

6 & Description \\
 & \begin{minipage}[t]{15cm}
{\footnotesize
\smallskip
Select Objects classified as point sources on 10 different coadd images
(including all bands). ~Evaluate the PSF model at the positions of these
Objects, and verify that subtracting a scaled version of the PSF model
from the coadd image yields residuals consistent with pure noise.

\medskip }
\end{minipage}
\\ \cdashline{2-2}


 & Expected Result \\
 & \begin{minipage}[t]{15cm}{\footnotesize
\smallskip
Images with the PSF model subtracted, leaving only residuals that are
consistent with being noise.

\medskip }
\end{minipage} \\ \cdashline{2-2}

 & Actual Result \\
 & \begin{minipage}[t]{15cm}{\footnotesize
\smallskip

\medskip }
\end{minipage} \\ \cdashline{2-2}

 & Status: \textbf{ Not Executed } \\ \hline

\end{longtable}

\paragraph{Test Case LVV-T132 - Verify implementation of Pre-cursor and Real Data
 }\mbox{}\\

Open  \href{https://jira.lsstcorp.org/secure/Tests.jspa#/testCase/LVV-T132}{\textit{ LVV-T132 } }
test case in Jira.

Demonstrate that pixel-oriented data from astronomical imaging cameras
(precursor or otherwise) can be processed using LSST Science Algorithms
and organized for access through the Data Butler Access Client. ~


\textbf{ Preconditions}:\\


Execution status: {\bf Not Executed }

Final comment:\\


Detailed steps results:

\begin{longtable}{p{1cm}p{15cm}}
\hline
{Step} & Step Details\\ \hline
1 & Description \\
 & \begin{minipage}[t]{15cm}
{\footnotesize
\smallskip
Confirm that the CI jobs used to test DRP and AP processing successfully
run. These jobs use precursor datasets from cameras other than LSST.

\medskip }
\end{minipage}
\\ \cdashline{2-2}


 & Expected Result \\
 & \begin{minipage}[t]{15cm}{\footnotesize
\smallskip

\medskip }
\end{minipage} \\ \cdashline{2-2}

 & Actual Result \\
 & \begin{minipage}[t]{15cm}{\footnotesize
\smallskip

\medskip }
\end{minipage} \\ \cdashline{2-2}

 & Status: \textbf{ Not Executed } \\ \hline

2 & Description \\
 & \begin{minipage}[t]{15cm}
{\footnotesize
\smallskip
For each of these two datasets, instantiate the Butler, ingest the data
products, and confirm that they exist as expected.

\medskip }
\end{minipage}
\\ \cdashline{2-2}


 & Expected Result \\
 & \begin{minipage}[t]{15cm}{\footnotesize
\smallskip
Processed images, catalogs, calibration information, and other related
data products are present and accessible via the Butler.

\medskip }
\end{minipage} \\ \cdashline{2-2}

 & Actual Result \\
 & \begin{minipage}[t]{15cm}{\footnotesize
\smallskip

\medskip }
\end{minipage} \\ \cdashline{2-2}

 & Status: \textbf{ Not Executed } \\ \hline

\end{longtable}

\paragraph{Test Case LVV-T40 - Verify implementation of Generate WCS for Visit Images
 }\mbox{}\\

Open  \href{https://jira.lsstcorp.org/secure/Tests.jspa#/testCase/LVV-T40}{\textit{ LVV-T40 } }
test case in Jira.

Verify that Processed Visit Images produced by the AP and DRP pipelines
include FITS WCS accurate to specified \textbf{astrometricAccuracy} over
the bounds of the image.


\textbf{ Preconditions}:\\


Execution status: {\bf Not Executed }

Final comment:\\


Detailed steps results:

\begin{longtable}{p{1cm}p{15cm}}
\hline
{Step} & Step Details\\ \hline
1 & Description \\
 & \begin{minipage}[t]{15cm}
{\footnotesize
\smallskip
Identify the path to the data repository, which we will refer to as
`DATA/path', then execute the following:

\medskip }
\end{minipage}
\\ \cdashline{2-2}


 & Expected Result \\
 & \begin{minipage}[t]{15cm}{\footnotesize
\smallskip
Butler repo available for reading.

\medskip }
\end{minipage} \\ \cdashline{2-2}

 & Actual Result \\
 & \begin{minipage}[t]{15cm}{\footnotesize
\smallskip

\medskip }
\end{minipage} \\ \cdashline{2-2}

 & Status: \textbf{ Not Executed } \\ \hline

2 & Description \\
 & \begin{minipage}[t]{15cm}
{\footnotesize
\smallskip
Ingest data from an appropriate processed dataset.

\medskip }
\end{minipage}
\\ \cdashline{2-2}


 & Expected Result \\
 & \begin{minipage}[t]{15cm}{\footnotesize
\smallskip

\medskip }
\end{minipage} \\ \cdashline{2-2}

 & Actual Result \\
 & \begin{minipage}[t]{15cm}{\footnotesize
\smallskip

\medskip }
\end{minipage} \\ \cdashline{2-2}

 & Status: \textbf{ Not Executed } \\ \hline

3 & Description \\
 & \begin{minipage}[t]{15cm}
{\footnotesize
\smallskip
Select a single visit from the dataset, and extract its WCS object and
the source list.

\medskip }
\end{minipage}
\\ \cdashline{2-2}


 & Expected Result \\
 & \begin{minipage}[t]{15cm}{\footnotesize
\smallskip
A table containing detected sources, and a WCS object associated with
that catalog.

\medskip }
\end{minipage} \\ \cdashline{2-2}

 & Actual Result \\
 & \begin{minipage}[t]{15cm}{\footnotesize
\smallskip

\medskip }
\end{minipage} \\ \cdashline{2-2}

 & Status: \textbf{ Not Executed } \\ \hline

4 & Description \\
 & \begin{minipage}[t]{15cm}
{\footnotesize
\smallskip
Confirm that each CCD within the visit image contains at
least~\textbf{astrometricMinStandards~}astrometric standards that were
used in deriving the astrometric solution.

\medskip }
\end{minipage}
\\ \cdashline{2-2}


 & Expected Result \\
 & \begin{minipage}[t]{15cm}{\footnotesize
\smallskip
At least \textbf{astrometricMinStandards} from each CCD\textbf{~}were
used in determining the WCS solution.

\medskip }
\end{minipage} \\ \cdashline{2-2}

 & Actual Result \\
 & \begin{minipage}[t]{15cm}{\footnotesize
\smallskip

\medskip }
\end{minipage} \\ \cdashline{2-2}

 & Status: \textbf{ Not Executed } \\ \hline

5 & Description \\
 & \begin{minipage}[t]{15cm}
{\footnotesize
\smallskip
Starting from the XY pixel coordinates of the sources, apply the WCS to
obtain RA, Dec coordinates.\\[2\baselineskip]

\medskip }
\end{minipage}
\\ \cdashline{2-2}


 & Expected Result \\
 & \begin{minipage}[t]{15cm}{\footnotesize
\smallskip
A list of RA, Dec coordinates for all sources in the catalog.

\medskip }
\end{minipage} \\ \cdashline{2-2}

 & Actual Result \\
 & \begin{minipage}[t]{15cm}{\footnotesize
\smallskip

\medskip }
\end{minipage} \\ \cdashline{2-2}

 & Status: \textbf{ Not Executed } \\ \hline

6 & Description \\
 & \begin{minipage}[t]{15cm}
{\footnotesize
\smallskip
We will assume that Gaia provides a source of ``truth.'' Match the
source list to Gaia DR2, and calculate the positional offset between the
test data and the Gaia catalog.

\medskip }
\end{minipage}
\\ \cdashline{2-2}


 & Expected Result \\
 & \begin{minipage}[t]{15cm}{\footnotesize
\smallskip
A matched catalog of sources in common between the test source list and
Gaia DR2.

\medskip }
\end{minipage} \\ \cdashline{2-2}

 & Actual Result \\
 & \begin{minipage}[t]{15cm}{\footnotesize
\smallskip

\medskip }
\end{minipage} \\ \cdashline{2-2}

 & Status: \textbf{ Not Executed } \\ \hline

7 & Description \\
 & \begin{minipage}[t]{15cm}
{\footnotesize
\smallskip
Apply appropriate cuts to extract the optimal dataset for comparison,
then calculate statistics (median, 1-sigma range, etc.; also plot a
histogram) of the offsets in milliarcseconds. Confirm that the offset is
less than \textbf{astrometricAccuracy}.

\medskip }
\end{minipage}
\\ \cdashline{2-2}


 & Expected Result \\
 & \begin{minipage}[t]{15cm}{\footnotesize
\smallskip
Histogram and relevant statistics needed to confirm that the WCS
transformation is~

\medskip }
\end{minipage} \\ \cdashline{2-2}

 & Actual Result \\
 & \begin{minipage}[t]{15cm}{\footnotesize
\smallskip

\medskip }
\end{minipage} \\ \cdashline{2-2}

 & Status: \textbf{ Not Executed } \\ \hline

8 & Description \\
 & \begin{minipage}[t]{15cm}
{\footnotesize
\smallskip
Repeat Step 5, but for subregions of the image, to confirm that the
accuracy criterion is met at all positions.

\medskip }
\end{minipage}
\\ \cdashline{2-2}


 & Expected Result \\
 & \begin{minipage}[t]{15cm}{\footnotesize
\smallskip
\textbf{astrometricAccuracy~}requirement is met over the entire image.

\medskip }
\end{minipage} \\ \cdashline{2-2}

 & Actual Result \\
 & \begin{minipage}[t]{15cm}{\footnotesize
\smallskip

\medskip }
\end{minipage} \\ \cdashline{2-2}

 & Status: \textbf{ Not Executed } \\ \hline

\end{longtable}

\paragraph{Test Case LVV-T1240 - Verify implementation of minimum astrometric standards per CCD
 }\mbox{}\\

Open  \href{https://jira.lsstcorp.org/secure/Tests.jspa#/testCase/LVV-T1240}{\textit{ LVV-T1240 } }
test case in Jira.

Verify that each CCD in a processed dataset had its astrometric solution
determined by at least~\textbf{astrometricMinStandards = 5~}astrometric
standards.


\textbf{ Preconditions}:\\


Execution status: {\bf Not Executed }

Final comment:\\


Detailed steps results:

\begin{longtable}{p{1cm}p{15cm}}
\hline
{Step} & Step Details\\ \hline
1 & Description \\
 & \begin{minipage}[t]{15cm}
{\footnotesize
\smallskip
Identify the path to the data repository, which we will refer to as
`DATA/path', then execute the following:

\medskip }
\end{minipage}
\\ \cdashline{2-2}


 & Expected Result \\
 & \begin{minipage}[t]{15cm}{\footnotesize
\smallskip
Butler repo available for reading.

\medskip }
\end{minipage} \\ \cdashline{2-2}

 & Actual Result \\
 & \begin{minipage}[t]{15cm}{\footnotesize
\smallskip

\medskip }
\end{minipage} \\ \cdashline{2-2}

 & Status: \textbf{ Not Executed } \\ \hline

2 & Description \\
 & \begin{minipage}[t]{15cm}
{\footnotesize
\smallskip
Ingest data from an appropriate processed dataset.

\medskip }
\end{minipage}
\\ \cdashline{2-2}


 & Expected Result \\
 & \begin{minipage}[t]{15cm}{\footnotesize
\smallskip

\medskip }
\end{minipage} \\ \cdashline{2-2}

 & Actual Result \\
 & \begin{minipage}[t]{15cm}{\footnotesize
\smallskip

\medskip }
\end{minipage} \\ \cdashline{2-2}

 & Status: \textbf{ Not Executed } \\ \hline

3 & Description \\
 & \begin{minipage}[t]{15cm}
{\footnotesize
\smallskip
Select a single visit from the dataset, and extract its calibration
data. For a subset of CCDs, check how many astrometric standards
contributed to the solution. Confirm that this number is at
least~\textbf{astrometricMinStandards = 5.}

\medskip }
\end{minipage}
\\ \cdashline{2-2}


 & Expected Result \\
 & \begin{minipage}[t]{15cm}{\footnotesize
\smallskip
At least \textbf{astrometricMinStandards} from each CCD\textbf{~}were
used in determining the WCS solution.

\medskip }
\end{minipage} \\ \cdashline{2-2}

 & Actual Result \\
 & \begin{minipage}[t]{15cm}{\footnotesize
\smallskip

\medskip }
\end{minipage} \\ \cdashline{2-2}

 & Status: \textbf{ Not Executed } \\ \hline

\end{longtable}

\paragraph{Test Case LVV-T376 - Verify the Calculation of Ellipticity Residuals and Correlations
 }\mbox{}\\

Open  \href{https://jira.lsstcorp.org/secure/Tests.jspa#/testCase/LVV-T376}{\textit{ LVV-T376 } }
test case in Jira.

Verify that the DMS includes software to enable the calculation of the
ellipticity residuals and correlation metrics defined in the OSS.~


\textbf{ Preconditions}:\\


Execution status: {\bf Not Executed }

Final comment:\\


Detailed steps results:

\begin{longtable}{p{1cm}p{15cm}}
\hline
{Step} & Step Details\\ \hline
1 & Description \\
 & \begin{minipage}[t]{15cm}
{\footnotesize
\smallskip
Identify the path to the data repository, which we will refer to as
`DATA/path', then execute the following:

\medskip }
\end{minipage}
\\ \cdashline{2-2}


 & Expected Result \\
 & \begin{minipage}[t]{15cm}{\footnotesize
\smallskip
Butler repo available for reading.

\medskip }
\end{minipage} \\ \cdashline{2-2}

 & Actual Result \\
 & \begin{minipage}[t]{15cm}{\footnotesize
\smallskip

\medskip }
\end{minipage} \\ \cdashline{2-2}

 & Status: \textbf{ Not Executed } \\ \hline

2 & Description \\
 & \begin{minipage}[t]{15cm}
{\footnotesize
\smallskip
Point the butler to an appropriate (precursor or simulated) dataset
containing data in all filters, that is sufficient for the purposes of
measuring astrometric performance metrics.

\medskip }
\end{minipage}
\\ \cdashline{2-2}


 & Expected Result \\
 & \begin{minipage}[t]{15cm}{\footnotesize
\smallskip

\medskip }
\end{minipage} \\ \cdashline{2-2}

 & Actual Result \\
 & \begin{minipage}[t]{15cm}{\footnotesize
\smallskip

\medskip }
\end{minipage} \\ \cdashline{2-2}

 & Status: \textbf{ Not Executed } \\ \hline

3 & Description \\
 & \begin{minipage}[t]{15cm}
{\footnotesize
\smallskip
Execute the LSST Stack package `validate\_drp` (or an alternate package
that is relevant) on this dataset to perform the measurements of the
metrics.

\medskip }
\end{minipage}
\\ \cdashline{2-2}


 & Expected Result \\
 & \begin{minipage}[t]{15cm}{\footnotesize
\smallskip
Measurements of validation metrics and the presence of QA plots
resulting from the validation pipeline.

\medskip }
\end{minipage} \\ \cdashline{2-2}

 & Actual Result \\
 & \begin{minipage}[t]{15cm}{\footnotesize
\smallskip

\medskip }
\end{minipage} \\ \cdashline{2-2}

 & Status: \textbf{ Not Executed } \\ \hline

4 & Description \\
 & \begin{minipage}[t]{15cm}
{\footnotesize
\smallskip
Compare measured ellipticity correlations to known (for simulated data)
or measured (if using precursor data) values from input (precursor or
simulated) data, and confirm that the output values for all of the
ellipticity performance metrics are as expected.

\medskip }
\end{minipage}
\\ \cdashline{2-2}


 & Expected Result \\
 & \begin{minipage}[t]{15cm}{\footnotesize
\smallskip
Measured ellipticity metrics that are within reasonable values given the
(known) input dataset.

\medskip }
\end{minipage} \\ \cdashline{2-2}

 & Actual Result \\
 & \begin{minipage}[t]{15cm}{\footnotesize
\smallskip

\medskip }
\end{minipage} \\ \cdashline{2-2}

 & Status: \textbf{ Not Executed } \\ \hline

\end{longtable}

\paragraph{Test Case LVV-T378 - Verify Calculation of Astrometric Performance Metrics
 }\mbox{}\\

Open  \href{https://jira.lsstcorp.org/secure/Tests.jspa#/testCase/LVV-T378}{\textit{ LVV-T378 } }
test case in Jira.

Verify that the DMS system provides software to calculate astrometric
performance metrics, and that the algorithms are properly calculating
the desired quantities. Note that because the DMS requirement is that
the software shall be provided (and not on the actual measured values of
the metrics), we verify all of the requirements via a single test case.


\textbf{ Preconditions}:\\


Execution status: {\bf Not Executed }

Final comment:\\


Detailed steps results:

\begin{longtable}{p{1cm}p{15cm}}
\hline
{Step} & Step Details\\ \hline
1 & Description \\
 & \begin{minipage}[t]{15cm}
{\footnotesize
\smallskip
Identify the path to the data repository, which we will refer to as
`DATA/path', then execute the following:

\medskip }
\end{minipage}
\\ \cdashline{2-2}


 & Expected Result \\
 & \begin{minipage}[t]{15cm}{\footnotesize
\smallskip
Butler repo available for reading.

\medskip }
\end{minipage} \\ \cdashline{2-2}

 & Actual Result \\
 & \begin{minipage}[t]{15cm}{\footnotesize
\smallskip

\medskip }
\end{minipage} \\ \cdashline{2-2}

 & Status: \textbf{ Not Executed } \\ \hline

2 & Description \\
 & \begin{minipage}[t]{15cm}
{\footnotesize
\smallskip
Point the butler to an appropriate (precursor or simulated) dataset
containing data in all filters, that is sufficient for the purposes of
measuring astrometric performance metrics.

\medskip }
\end{minipage}
\\ \cdashline{2-2}


 & Expected Result \\
 & \begin{minipage}[t]{15cm}{\footnotesize
\smallskip

\medskip }
\end{minipage} \\ \cdashline{2-2}

 & Actual Result \\
 & \begin{minipage}[t]{15cm}{\footnotesize
\smallskip

\medskip }
\end{minipage} \\ \cdashline{2-2}

 & Status: \textbf{ Not Executed } \\ \hline

3 & Description \\
 & \begin{minipage}[t]{15cm}
{\footnotesize
\smallskip
Execute the LSST Stack package `validate\_drp` (or an alternate package
that is relevant) on this dataset to perform the measurements of the
metrics.

\medskip }
\end{minipage}
\\ \cdashline{2-2}


 & Expected Result \\
 & \begin{minipage}[t]{15cm}{\footnotesize
\smallskip
Measurements of validation metrics and the presence of QA plots
resulting from the validation pipeline.

\medskip }
\end{minipage} \\ \cdashline{2-2}

 & Actual Result \\
 & \begin{minipage}[t]{15cm}{\footnotesize
\smallskip

\medskip }
\end{minipage} \\ \cdashline{2-2}

 & Status: \textbf{ Not Executed } \\ \hline

4 & Description \\
 & \begin{minipage}[t]{15cm}
{\footnotesize
\smallskip
Compare measured astrometry to known (for simulated data) or measured
(if using precursor data) values from input (precursor or simulated)
data, and confirm that the output values for all of the astrometric
performance metrics are as expected.

\medskip }
\end{minipage}
\\ \cdashline{2-2}


 & Expected Result \\
 & \begin{minipage}[t]{15cm}{\footnotesize
\smallskip
Measured astrometry metrics that are within reasonable values given the
(known) input dataset.

\medskip }
\end{minipage} \\ \cdashline{2-2}

 & Actual Result \\
 & \begin{minipage}[t]{15cm}{\footnotesize
\smallskip

\medskip }
\end{minipage} \\ \cdashline{2-2}

 & Status: \textbf{ Not Executed } \\ \hline

\end{longtable}

\paragraph{Test Case LVV-T377 - Verify Calculation of Photometric Performance Metrics
 }\mbox{}\\

Open  \href{https://jira.lsstcorp.org/secure/Tests.jspa#/testCase/LVV-T377}{\textit{ LVV-T377 } }
test case in Jira.

Verify that the DMS system provides software to calculate photometric
performance metrics, and that the algorithms are properly calculating
the desired quantities. Note that because the DMS requirement is that
the software shall be provided (and not on the actual measured values of
the metrics), we verify all of the requirements via a single test case.


\textbf{ Preconditions}:\\


Execution status: {\bf Not Executed }

Final comment:\\


Detailed steps results:

\begin{longtable}{p{1cm}p{15cm}}
\hline
{Step} & Step Details\\ \hline
1 & Description \\
 & \begin{minipage}[t]{15cm}
{\footnotesize
\smallskip
Identify the path to the data repository, which we will refer to as
`DATA/path', then execute the following:

\medskip }
\end{minipage}
\\ \cdashline{2-2}


 & Expected Result \\
 & \begin{minipage}[t]{15cm}{\footnotesize
\smallskip
Butler repo available for reading.

\medskip }
\end{minipage} \\ \cdashline{2-2}

 & Actual Result \\
 & \begin{minipage}[t]{15cm}{\footnotesize
\smallskip

\medskip }
\end{minipage} \\ \cdashline{2-2}

 & Status: \textbf{ Not Executed } \\ \hline

2 & Description \\
 & \begin{minipage}[t]{15cm}
{\footnotesize
\smallskip
Point the butler to a simulated dataset containing data in all filters,
that is sufficient for the purposes of measuring photometric performance
metrics.

\medskip }
\end{minipage}
\\ \cdashline{2-2}


 & Expected Result \\
 & \begin{minipage}[t]{15cm}{\footnotesize
\smallskip

\medskip }
\end{minipage} \\ \cdashline{2-2}

 & Actual Result \\
 & \begin{minipage}[t]{15cm}{\footnotesize
\smallskip

\medskip }
\end{minipage} \\ \cdashline{2-2}

 & Status: \textbf{ Not Executed } \\ \hline

3 & Description \\
 & \begin{minipage}[t]{15cm}
{\footnotesize
\smallskip
Execute the LSST Stack package `validate\_drp` (or an alternate package
that is relevant) on this dataset to perform the measurements of the
metrics.

\medskip }
\end{minipage}
\\ \cdashline{2-2}


 & Expected Result \\
 & \begin{minipage}[t]{15cm}{\footnotesize
\smallskip
Measurements of validation metrics and the presence of QA plots
resulting from the validation pipeline.

\medskip }
\end{minipage} \\ \cdashline{2-2}

 & Actual Result \\
 & \begin{minipage}[t]{15cm}{\footnotesize
\smallskip

\medskip }
\end{minipage} \\ \cdashline{2-2}

 & Status: \textbf{ Not Executed } \\ \hline

4 & Description \\
 & \begin{minipage}[t]{15cm}
{\footnotesize
\smallskip
Compare measured photometry to known values from input simulated data,
and confirm that the output values for all of the photometric
performance metrics are as expected.

\medskip }
\end{minipage}
\\ \cdashline{2-2}


 & Expected Result \\
 & \begin{minipage}[t]{15cm}{\footnotesize
\smallskip
Measured astrometry metrics that are within reasonable values given the
(known) input dataset.

\medskip }
\end{minipage} \\ \cdashline{2-2}

 & Actual Result \\
 & \begin{minipage}[t]{15cm}{\footnotesize
\smallskip

\medskip }
\end{minipage} \\ \cdashline{2-2}

 & Status: \textbf{ Not Executed } \\ \hline

\end{longtable}

\paragraph{Test Case LVV-T28 - Verify implementation of Measurements in catalogs
 }\mbox{}\\

Open  \href{https://jira.lsstcorp.org/secure/Tests.jspa#/testCase/LVV-T28}{\textit{ LVV-T28 } }
test case in Jira.

Verify that source measurements in catalogs are in flux units.


\textbf{ Preconditions}:\\


Execution status: {\bf Not Executed }

Final comment:\\


Detailed steps results:

\begin{longtable}{p{1cm}p{15cm}}
\hline
{Step} & Step Details\\ \hline
1 & Description \\
 & \begin{minipage}[t]{15cm}
{\footnotesize
\smallskip
The DM Stack and Alert Processing packaged shall be initialized as
described in LVT-T17 (AG-00-00).

\medskip }
\end{minipage}
\\ \cdashline{2-2}


 & Expected Result \\
 & \begin{minipage}[t]{15cm}{\footnotesize
\smallskip

\medskip }
\end{minipage} \\ \cdashline{2-2}

 & Actual Result \\
 & \begin{minipage}[t]{15cm}{\footnotesize
\smallskip

\medskip }
\end{minipage} \\ \cdashline{2-2}

 & Status: \textbf{ Not Executed } \\ \hline

2 & Description \\
 & \begin{minipage}[t]{15cm}
{\footnotesize
\smallskip
The alert generation processing will be executed using the verification
cluster:\\[2\baselineskip]```bash\\
python ap\_verify/bin/prepare\_demo\_slurm\_files.py\\
\# At present we must run a single ccd+visit to handle ingestion
before\\
\# parallel processing can begin\\
./ap\_verify/bin/exec\_demo\_run\_1ccd.sh 410915 25\\
ln -s ap\_verify/bin/demo\_run.sl\\
ln -s ap\_verify/bin/demo\_cmds.conf\\
sbatch demo\_run.sl\\
```\\[2\baselineskip]and any errors or failures reported.

\medskip }
\end{minipage}
\\ \cdashline{2-2}


 & Expected Result \\
 & \begin{minipage}[t]{15cm}{\footnotesize
\smallskip

\medskip }
\end{minipage} \\ \cdashline{2-2}

 & Actual Result \\
 & \begin{minipage}[t]{15cm}{\footnotesize
\smallskip

\medskip }
\end{minipage} \\ \cdashline{2-2}

 & Status: \textbf{ Not Executed } \\ \hline

3 & Description \\
 & \begin{minipage}[t]{15cm}
{\footnotesize
\smallskip
A ``Data Butler'' will be initialized to access the repository.

\medskip }
\end{minipage}
\\ \cdashline{2-2}


 & Expected Result \\
 & \begin{minipage}[t]{15cm}{\footnotesize
\smallskip

\medskip }
\end{minipage} \\ \cdashline{2-2}

 & Actual Result \\
 & \begin{minipage}[t]{15cm}{\footnotesize
\smallskip

\medskip }
\end{minipage} \\ \cdashline{2-2}

 & Status: \textbf{ Not Executed } \\ \hline

4 & Description \\
 & \begin{minipage}[t]{15cm}
{\footnotesize
\smallskip
For each of the expected data products types (listed in §4.2.2) and each
of the expected units (PVIs, catalogs, etc.), the data product will be
retrieved from the Butler and verified to be non-empty.

\medskip }
\end{minipage}
\\ \cdashline{2-2}


 & Expected Result \\
 & \begin{minipage}[t]{15cm}{\footnotesize
\smallskip

\medskip }
\end{minipage} \\ \cdashline{2-2}

 & Actual Result \\
 & \begin{minipage}[t]{15cm}{\footnotesize
\smallskip

\medskip }
\end{minipage} \\ \cdashline{2-2}

 & Status: \textbf{ Not Executed } \\ \hline

5 & Description \\
 & \begin{minipage}[t]{15cm}
{\footnotesize
\smallskip
DIAObjects are currently only stored in a database, without shims to the
Butler, so the existence of the database table and its non-empty
contents will be verified by directly accessing it using sqlite3 and
executing appropriate SQL queries.

\medskip }
\end{minipage}
\\ \cdashline{2-2}


 & Expected Result \\
 & \begin{minipage}[t]{15cm}{\footnotesize
\smallskip

\medskip }
\end{minipage} \\ \cdashline{2-2}

 & Actual Result \\
 & \begin{minipage}[t]{15cm}{\footnotesize
\smallskip

\medskip }
\end{minipage} \\ \cdashline{2-2}

 & Status: \textbf{ Not Executed } \\ \hline

6 & Description \\
 & \begin{minipage}[t]{15cm}
{\footnotesize
\smallskip
The DM Stack shall be initialized using the loadLSST script (as
described in DRP-00-00).

\medskip }
\end{minipage}
\\ \cdashline{2-2}


 & Expected Result \\
 & \begin{minipage}[t]{15cm}{\footnotesize
\smallskip

\medskip }
\end{minipage} \\ \cdashline{2-2}

 & Actual Result \\
 & \begin{minipage}[t]{15cm}{\footnotesize
\smallskip

\medskip }
\end{minipage} \\ \cdashline{2-2}

 & Status: \textbf{ Not Executed } \\ \hline

7 & Description \\
 & \begin{minipage}[t]{15cm}
{\footnotesize
\smallskip
A ``Data Butler'' will be initialized to access the repository.

\medskip }
\end{minipage}
\\ \cdashline{2-2}


 & Expected Result \\
 & \begin{minipage}[t]{15cm}{\footnotesize
\smallskip

\medskip }
\end{minipage} \\ \cdashline{2-2}

 & Actual Result \\
 & \begin{minipage}[t]{15cm}{\footnotesize
\smallskip

\medskip }
\end{minipage} \\ \cdashline{2-2}

 & Status: \textbf{ Not Executed } \\ \hline

8 & Description \\
 & \begin{minipage}[t]{15cm}
{\footnotesize
\smallskip
For each of the expected data products types (listed in Test Items
section §4.3.2) and each of the expected units (PVIs, coadds, etc), the
data product will be retrieved from the Butler and verified to be
non-empty.

\medskip }
\end{minipage}
\\ \cdashline{2-2}


 & Expected Result \\
 & \begin{minipage}[t]{15cm}{\footnotesize
\smallskip

\medskip }
\end{minipage} \\ \cdashline{2-2}

 & Actual Result \\
 & \begin{minipage}[t]{15cm}{\footnotesize
\smallskip

\medskip }
\end{minipage} \\ \cdashline{2-2}

 & Status: \textbf{ Not Executed } \\ \hline

9 & Description \\
 & \begin{minipage}[t]{15cm}
{\footnotesize
\smallskip
Identify the path to the data repository, which we will refer to as
`DATA/path', then execute the following:

\medskip }
\end{minipage}
\\ \cdashline{2-2}


 & Expected Result \\
 & \begin{minipage}[t]{15cm}{\footnotesize
\smallskip
Butler repo available for reading.

\medskip }
\end{minipage} \\ \cdashline{2-2}

 & Actual Result \\
 & \begin{minipage}[t]{15cm}{\footnotesize
\smallskip

\medskip }
\end{minipage} \\ \cdashline{2-2}

 & Status: \textbf{ Not Executed } \\ \hline

10 & Description \\
 & \begin{minipage}[t]{15cm}
{\footnotesize
\smallskip
Identify and read appropriate processed precursor datasets with the
Butler, including one containing single-visit images, one with coadds,
and one with difference imaging.~

\medskip }
\end{minipage}
\\ \cdashline{2-2}


 & Expected Result \\
 & \begin{minipage}[t]{15cm}{\footnotesize
\smallskip

\medskip }
\end{minipage} \\ \cdashline{2-2}

 & Actual Result \\
 & \begin{minipage}[t]{15cm}{\footnotesize
\smallskip

\medskip }
\end{minipage} \\ \cdashline{2-2}

 & Status: \textbf{ Not Executed } \\ \hline

11 & Description \\
 & \begin{minipage}[t]{15cm}
{\footnotesize
\smallskip
Verify that each of the single-visit, coadd, and difference image
catalogs provide measurements in flux units.

\medskip }
\end{minipage}
\\ \cdashline{2-2}


 & Expected Result \\
 & \begin{minipage}[t]{15cm}{\footnotesize
\smallskip
Confirmation of measurements in catalogs encoded in flux units.

\medskip }
\end{minipage} \\ \cdashline{2-2}

 & Actual Result \\
 & \begin{minipage}[t]{15cm}{\footnotesize
\smallskip

\medskip }
\end{minipage} \\ \cdashline{2-2}

 & Status: \textbf{ Not Executed } \\ \hline

\end{longtable}

\paragraph{Test Case LVV-T43 - Verify implementation of Background Model Calculation
 }\mbox{}\\

Open  \href{https://jira.lsstcorp.org/secure/Tests.jspa#/testCase/LVV-T43}{\textit{ LVV-T43 } }
test case in Jira.

Verify that Processed Visit Images produced by the DRP and AP pipelines
have had a model of the background subtracted, and that this model is
persisted in a way that permits the background subtracted from any CCD
to be retrieved along with the image for that CCD.


\textbf{ Preconditions}:\\


Execution status: {\bf Not Executed }

Final comment:\\


Detailed steps results:

\begin{longtable}{p{1cm}p{15cm}}
\hline
{Step} & Step Details\\ \hline
1 & Description \\
 & \begin{minipage}[t]{15cm}
{\footnotesize
\smallskip
Identify a dataset with processed visit images in multiple filters.

\medskip }
\end{minipage}
\\ \cdashline{2-2}


 & Expected Result \\
 & \begin{minipage}[t]{15cm}{\footnotesize
\smallskip

\medskip }
\end{minipage} \\ \cdashline{2-2}

 & Actual Result \\
 & \begin{minipage}[t]{15cm}{\footnotesize
\smallskip

\medskip }
\end{minipage} \\ \cdashline{2-2}

 & Status: \textbf{ Not Executed } \\ \hline

2 & Description \\
 & \begin{minipage}[t]{15cm}
{\footnotesize
\smallskip
Identify the path to the data repository, which we will refer to as
`DATA/path', then execute the following:

\medskip }
\end{minipage}
\\ \cdashline{2-2}


 & Expected Result \\
 & \begin{minipage}[t]{15cm}{\footnotesize
\smallskip
Butler repo available for reading.

\medskip }
\end{minipage} \\ \cdashline{2-2}

 & Actual Result \\
 & \begin{minipage}[t]{15cm}{\footnotesize
\smallskip

\medskip }
\end{minipage} \\ \cdashline{2-2}

 & Status: \textbf{ Not Executed } \\ \hline

3 & Description \\
 & \begin{minipage}[t]{15cm}
{\footnotesize
\smallskip
Display an image of the background model for a full CCD. Repeat this for
all available filters, and confirm that the background is smoothly
varying and defined over the full CCD.

\medskip }
\end{minipage}
\\ \cdashline{2-2}


 & Expected Result \\
 & \begin{minipage}[t]{15cm}{\footnotesize
\smallskip
Well-formed background covering the entire CCD for all CCDs in all
filters.

\medskip }
\end{minipage} \\ \cdashline{2-2}

 & Actual Result \\
 & \begin{minipage}[t]{15cm}{\footnotesize
\smallskip

\medskip }
\end{minipage} \\ \cdashline{2-2}

 & Status: \textbf{ Not Executed } \\ \hline

\end{longtable}


\newpage
\appendix
%Make sure lsst-texmf/bin/generateAcronyms.py is in your path
\section{Acronyms used in this document}\label{sec:acronyms}
\addtocounter{table}{-1}
\begin{longtable}{p{0.145\textwidth}p{0.8\textwidth}}\hline
\textbf{Acronym} & \textbf{Description}  \\\hline

AP & Alert Production \\\hline
CCD & Charge-Coupled Device \\\hline
CI & Cyber Infrastructure \\\hline
DM & Data Management \\\hline
DMS & Data Management Subsystem \\\hline
DMSR & DM System Requirements; LSE-61 \\\hline
DMTR & DM Test Report \\\hline
DRP & Data Release Production \\\hline
FITS & Flexible Image Transport System \\\hline
LDF & LSST Data Facility \\\hline
LDM & LSST Data Management (Document Handle) \\\hline
LSE & LSST Systems Engineering (Document Handle) \\\hline
LSP & LSST Science Platform \\\hline
LSST & Large Synoptic Survey Telescope \\\hline
NCSA & National Center for Supercomputing Applications \\\hline
OSS & Observatory System Specifications; LSE-30 \\\hline
PSF & Point Spread Function \\\hline
QA & Quality Assurance \\\hline
RA & Right Ascension \\\hline
SQL & Structured Query Language \\\hline
WCS & World Coordinate System \\\hline
\end{longtable}


\end{document}
